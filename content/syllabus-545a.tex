% Options for packages loaded elsewhere
\PassOptionsToPackage{unicode}{hyperref}
\PassOptionsToPackage{hyphens}{url}
%
\documentclass[
]{article}
\usepackage{lmodern}
\usepackage{amssymb,amsmath}
\usepackage{ifxetex,ifluatex}
\ifnum 0\ifxetex 1\fi\ifluatex 1\fi=0 % if pdftex
  \usepackage[T1]{fontenc}
  \usepackage[utf8]{inputenc}
  \usepackage{textcomp} % provide euro and other symbols
\else % if luatex or xetex
  \usepackage{unicode-math}
  \defaultfontfeatures{Scale=MatchLowercase}
  \defaultfontfeatures[\rmfamily]{Ligatures=TeX,Scale=1}
\fi
% Use upquote if available, for straight quotes in verbatim environments
\IfFileExists{upquote.sty}{\usepackage{upquote}}{}
\IfFileExists{microtype.sty}{% use microtype if available
  \usepackage[]{microtype}
  \UseMicrotypeSet[protrusion]{basicmath} % disable protrusion for tt fonts
}{}
\makeatletter
\@ifundefined{KOMAClassName}{% if non-KOMA class
  \IfFileExists{parskip.sty}{%
    \usepackage{parskip}
  }{% else
    \setlength{\parindent}{0pt}
    \setlength{\parskip}{6pt plus 2pt minus 1pt}}
}{% if KOMA class
  \KOMAoptions{parskip=half}}
\makeatother
\usepackage{xcolor}
\IfFileExists{xurl.sty}{\usepackage{xurl}}{} % add URL line breaks if available
\IfFileExists{bookmark.sty}{\usepackage{bookmark}}{\usepackage{hyperref}}
\hypersetup{
  pdftitle={STAT 545A Course syllabus: 2020/2021},
  hidelinks,
  pdfcreator={LaTeX via pandoc}}
\urlstyle{same} % disable monospaced font for URLs
\usepackage[margin=1in]{geometry}
\usepackage{longtable,booktabs}
% Correct order of tables after \paragraph or \subparagraph
\usepackage{etoolbox}
\makeatletter
\patchcmd\longtable{\par}{\if@noskipsec\mbox{}\fi\par}{}{}
\makeatother
% Allow footnotes in longtable head/foot
\IfFileExists{footnotehyper.sty}{\usepackage{footnotehyper}}{\usepackage{footnote}}
\makesavenoteenv{longtable}
\usepackage{graphicx,grffile}
\makeatletter
\def\maxwidth{\ifdim\Gin@nat@width>\linewidth\linewidth\else\Gin@nat@width\fi}
\def\maxheight{\ifdim\Gin@nat@height>\textheight\textheight\else\Gin@nat@height\fi}
\makeatother
% Scale images if necessary, so that they will not overflow the page
% margins by default, and it is still possible to overwrite the defaults
% using explicit options in \includegraphics[width, height, ...]{}
\setkeys{Gin}{width=\maxwidth,height=\maxheight,keepaspectratio}
% Set default figure placement to htbp
\makeatletter
\def\fps@figure{htbp}
\makeatother
\setlength{\emergencystretch}{3em} % prevent overfull lines
\providecommand{\tightlist}{%
  \setlength{\itemsep}{0pt}\setlength{\parskip}{0pt}}
\setcounter{secnumdepth}{5}

\title{STAT 545A Course syllabus: 2020/2021}
\author{}
\date{\vspace{-2.5em}}

\begin{document}
\maketitle

{
\setcounter{tocdepth}{2}
\tableofcontents
}
How to make a clean and modern data analysis, Part I.

\textbf{Website}: \url{https://stat545.stat.ubc.ca/}

\textbf{Duration}: 2020-09-08 - 2020-10-23

\begin{itemize}
\tightlist
\item
  Introduction to \href{http://www.r-project.org}{R} and the
  \href{http://www.rstudio.com/products/rstudio/}{RStudio IDE}: scripts,
  the workspace, RStudio Projects, daily workflow
\item
  Generate reports from R scripts and
  \href{http://rmarkdown.rstudio.com}{R Markdown}
\item
  Coding style, file and project organization
\item
  Data frames or ``tibbles'' are the core data structure for data
  analysis: care for them with the tidyverse
\item
  Data visualization with \href{http://ggplot2.org}{\texttt{ggplot2}}
\item
  How to write functions and work with R in a functional style
\item
  Version control with Git; collaboration via
  \href{https://github.com}{GitHub}
\end{itemize}

\hypertarget{teaching-team}{%
\section{Teaching Team}\label{teaching-team}}

Instructor: Dr.~Vincenzo Coia

Teaching Assistants:

\begin{itemize}
\tightlist
\item
  Almas Khan
\item
  Icíar Fernández Boyano
\item
  Diana Lin
\item
  Victor Yuan
\end{itemize}

\hypertarget{lectures}{%
\section{Lectures}\label{lectures}}

Tuesdays and Thursdays 0930-1100 PST

There will always be two TA's in class to help students with the live
coding exercises.

\begin{longtable}[]{@{}rlll@{}}
\toprule
Lesson & Weekday & Date & Topic\tabularnewline
\midrule
\endhead
1 & Thu & Sep 10 & Introduction to STAT545 and R\tabularnewline
2 & Tue & Sep 15 & Collaboration and Version Control\tabularnewline
3 & Thu & Sep 17 & R Markdown and Reproducibility\tabularnewline
4 & Tue & Sep 22 & Data Wrangling Part I\tabularnewline
5 & Thu & Sep 24 & Plotting Part I\tabularnewline
6 & Tue & Sep 29 & Data Wrangling Part II\tabularnewline
7 & Thu & Oct 1 & Plotting Part II\tabularnewline
8 & Tue & Oct 6 & Tidy data\tabularnewline
9 & Thu & Oct 8 & The model-fitting paradigm in R\tabularnewline
10 & Tue & Oct 13 & Special data types: factors and dates\tabularnewline
11 & Thu & Oct 15 & Tibble joins\tabularnewline
12 & Tue & Oct 20 & File input/output\tabularnewline
13 & Thu & Oct 22 & Choose your own adventure\tabularnewline
\bottomrule
\end{longtable}

\textbf{What to expect during class}

Lectures will consist of the following components, in this order:

\begin{enumerate}
\def\labelenumi{\arabic{enumi}.}
\tightlist
\item
  Live demo and questions (20-30 minutes)

  \begin{itemize}
  \tightlist
  \item
    Instructor demonstrates techniques; students ask questions and pose
    challenges.
  \item
    Discussion in a chat are processed by a TA: answer directly, unless
    the question is more involved, in which case it enters a queue
    (managed by that TA).
  \end{itemize}
\item
  Quiz

  \begin{itemize}
  \tightlist
  \item
    Only graded by your participation, not correctness. You have 24
    hours to complete.
  \item
    We discuss in class.
  \end{itemize}
\item
  Break: 5 minutes

  \begin{itemize}
  \tightlist
  \item
    We'll try playing
    \href{https://www.youtube.com/playlist?list=PL6JWPcX0WhHU97aqiRA4P6Jta9f04Nanm}{Random
    Acts of Exercise} (Each body is different, so make this your
    practice by modifying the activity based on what your body can do.)
  \end{itemize}
\item
  Lab (45 minutes):

  \begin{itemize}
  \tightlist
  \item
    Instructor introduces the worksheet exercises for the period
  \item
    Instructor + 1 TA goes through random groups of 3 or 4 students in
    turn, coming in with some discussion points, but also allowing a
    natural conversation to unfold about the worksheet, as well as
    perhaps how the course is going.
  \item
    At the same time, a TA remains present to answer questions.
  \end{itemize}
\end{enumerate}

\hypertarget{deliverables}{%
\section{Deliverables}\label{deliverables}}

\begin{longtable}[]{@{}llll@{}}
\toprule
\begin{minipage}[b]{0.25\columnwidth}\raggedright
Deliverable\strut
\end{minipage} & \begin{minipage}[b]{0.28\columnwidth}\raggedright
Submission Frequency\strut
\end{minipage} & \begin{minipage}[b]{0.19\columnwidth}\raggedright
Percent Grade\strut
\end{minipage} & \begin{minipage}[b]{0.16\columnwidth}\raggedright
Description\strut
\end{minipage}\tabularnewline
\midrule
\endhead
\begin{minipage}[t]{0.25\columnwidth}\raggedright
Class worksheets (5)\strut
\end{minipage} & \begin{minipage}[t]{0.28\columnwidth}\raggedright
weekly\strut
\end{minipage} & \begin{minipage}[t]{0.19\columnwidth}\raggedright
10\strut
\end{minipage} & \begin{minipage}[t]{0.16\columnwidth}\raggedright
Autograded walkthroughs to guide student learning.\strut
\end{minipage}\tabularnewline
\begin{minipage}[t]{0.25\columnwidth}\raggedright
Participation quizzes (13)\strut
\end{minipage} & \begin{minipage}[t]{0.28\columnwidth}\raggedright
every class\strut
\end{minipage} & \begin{minipage}[t]{0.19\columnwidth}\raggedright
5\strut
\end{minipage} & \begin{minipage}[t]{0.16\columnwidth}\raggedright
Get full marks by answering a few quick questions per class -- doesn't
matter if you're right or wrong! 24h submission period.\strut
\end{minipage}\tabularnewline
\begin{minipage}[t]{0.25\columnwidth}\raggedright
Mini data analysis\strut
\end{minipage} & \begin{minipage}[t]{0.28\columnwidth}\raggedright
three checkpoints\strut
\end{minipage} & \begin{minipage}[t]{0.19\columnwidth}\raggedright
50\strut
\end{minipage} & \begin{minipage}[t]{0.16\columnwidth}\raggedright
Students write their own mini data analysis.\strut
\end{minipage}\tabularnewline
\begin{minipage}[t]{0.25\columnwidth}\raggedright
Collaborative Troubleshooting project\strut
\end{minipage} & \begin{minipage}[t]{0.28\columnwidth}\raggedright
three checkpoints\strut
\end{minipage} & \begin{minipage}[t]{0.19\columnwidth}\raggedright
35\strut
\end{minipage} & \begin{minipage}[t]{0.16\columnwidth}\raggedright
Team project intended for practicing version control and collaboration,
by answering some debugging problems.\strut
\end{minipage}\tabularnewline
\bottomrule
\end{longtable}

More details can be found \href{https://stat545.stat.ubc.ca/course}{on
the course dashboard}.

\hypertarget{auditing-students}{%
\section{Auditing Students}\label{auditing-students}}

Auditing students are expected to complete all assessments (assignments,
peer reviews, and participation). The difference between enrolling for
credit is that auditing students are graded on each assignment on a
pass/fail basis.

\hypertarget{privacy}{%
\section{Privacy}\label{privacy}}

\hypertarget{slack}{%
\subsection{Slack}\label{slack}}

STAT 545 uses Slack for informal communications. Note that the messages
sent on Slack are stored on a US server.

\hypertarget{github.com}{%
\subsection{GitHub.com}\label{github.com}}

STAT 545 asks students to work on github.com. Please produce work
knowing that the material you put on GitHub will be stored on US
servers.

\hypertarget{policies}{%
\section{Policies}\label{policies}}

In addition to
\href{http://www.calendar.ubc.ca/vancouver/?tree=3,0,0,0}{UBC's
Campus-wide Policies and Regulations}, STAT 545A and STAT 547M adopt the
following policies.

\hypertarget{communications}{%
\subsection{Communications}\label{communications}}

The teaching team can't guarantee that they will be able to respond to
student messages outside of typical workday hours (0900-1700 PST). So,
please be mindful of a \textbf{17:00 PST cutoff on Fridays} when asking
assignment-related questions.

Please read \href{/slack_communication}{this} before messaging the
teaching team.

\hypertarget{late-policy}{%
\subsection{Late Policy}\label{late-policy}}

A late submission is defined as any work, including quizzes, submitted
after the deadline. For a late submission, the student will receive a
50\% scaling of their grade for the first occurrence, and will receive a
grade of 0 for subsequent occurrences. In all cases, late submissions
past 24 hours will receive a zero.

\hypertarget{academic-concession}{%
\subsection{Academic Concession}\label{academic-concession}}

UBC no longer requires a doctor's note (or supporting documentation) for
\href{http://www.calendar.ubc.ca/vancouver/index.cfm?tree=3,48,0,0}{academic
concession}. A self-declaration will suffice --
\href{/concession_template.pdf}{here} is a template you can use. Please
submit this to the instructor.

For this course, a ``conflicting responsibility'' includes needing to
travel for a conference or field work.

If you arrange to have an assignment submitted late, you may not be able
to receive feedback from your peers.

\hypertarget{plagiarism}{%
\subsection{Plagiarism}\label{plagiarism}}

Plagiarism, which is intellectual theft, occurs where an individual
submits or presents the oral or written work of another person as his or
her own and can include:

\begin{itemize}
\tightlist
\item
  multiple students submitting the same response
\item
  copying from sources without citing them
\item
  copying verbatim (word-for-word) from source and citing, but failing
  to make it explicit that this is a quotation (quotations should be
  used only rarely, if at all)
\end{itemize}

Plagiarism will not be tolerated in the MDS program and may result in
dismissal from the program. Students are responsible for ensuring that
any work submitted does not constitute plagiarism. Students who are in
any doubt as to what constitutes plagiarism should consult their
Instructor before handing in any assignments.

For more information see the
\href{http://www.calendar.ubc.ca/vancouver/index.cfm?tree=3,54,111,959}{UBC
Academic Misconduct policies}.

\hypertarget{code-plagiarism}{%
\subsubsection{Code Plagiarism}\label{code-plagiarism}}

Students must correctly cite any code that has been authored by someone
else or by the student themselves for other assignments. Cases of code
plagiarism may include, but are not limited to:

\begin{itemize}
\tightlist
\item
  the reproduction (copying and pasting) of code with none or minimal
  reformatting (e.g., changing the name of the variables)
\item
  the translation of an algorithm or a script from a language to another
\item
  the generation of code by automatic code-generations software
\end{itemize}

An ``adequate acknowledgement'' requires a detailed identification of
the (parts of the) code reused and a full citation of the original
source code that has been reused.

\hypertarget{video-conferencing-with-zoom}{%
\subsection{Video Conferencing with
Zoom}\label{video-conferencing-with-zoom}}

Students are encouraged to turn their cameras on whenever they are using
Zoom. However, we understand that this can be an issue for some people.
As such, you will never be expected to turn your camera on.

\hypertarget{ubcs-policies-and-resources-to-support-student-success}{%
\subsection{UBC's Policies and Resources to Support Student
Success}\label{ubcs-policies-and-resources-to-support-student-success}}

UBC provides resources to support student learning and to maintain
healthy lifestyles but recognizes that sometimes crises arise and so
there are additional resources to access including those for survivors
of sexual violence. UBC values respect for the person and ideas of all
members of the academic community. Harassment and discrimination are not
tolerated nor is suppression of academic freedom. UBC provides
appropriate accommodation for students with disabilities and for
religious, spiritual and cultural observances. UBC values academic
honesty and students ae expected to acknowledge the ideas generated by
others and to uphold the highest academic standards in all of their
actions. Details of the policies and how to access support are available
\href{https://senate.ubc.ca/policies-resources-support-student-success}{here}.

\hypertarget{potential-restrictions-to-international-students-online-learning-experiences-as-a-result-of-remote-learning}{%
\subsection{Potential restrictions to international students' online
learning experiences as a result of remote
learning}\label{potential-restrictions-to-international-students-online-learning-experiences-as-a-result-of-remote-learning}}

During this pandemic, the shift to online learning has greatly altered
teaching and studying at UBC, including changes to health and safety
considerations. Keep in mind that some UBC courses might cover topics
that are censored or considered illegal by non-Canadian governments.
This may include, but is not limited to, human rights, representative
government, defamation, obscenity, gender or sexuality, and historical
or current geopolitical controversies. If you are a student living
abroad, you will be subject to the laws of your local jurisdiction, and
your local authorities might limit your access to course material or
take punitive action against you. UBC is strongly committed to academic
freedom, but has no control over foreign authorities (please visit
\url{http://www.calendar.ubc.ca/vancouver/index.cfm?tree=3,33,86,0} for
an articulation of the values of the University conveyed in the Senate
Statement on Academic Freedom). Thus, we recognize that students will
have legitimate reason to exercise caution in studying certain subjects.
If you have concerns regarding your personal situation, consider
postponing taking a course with manifest risks, until you are back on
campus or reach out to your academic advisor to find substitute courses.
For further information and support, please visit:
\url{http://academic.ubc.ca/support-resources/freedom-expression}

\end{document}
